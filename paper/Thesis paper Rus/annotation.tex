\newpage


\begin{abstract}
В данной работе исследуется способ уменьшения размерности пространства обучаемых параметров в задаче детектирования ai текстов(задача многоклассовой классификации). Для fine tuning использовалась модель RoBerta с LoRA адаптером. Было проведено несколько экспериментов, чтобы выяснить, является ли использование LoRA для аппроксимации матрицы весов эффективным с точки зрения времени, ресурсов или точности. Было показано, что при меньших ресурсах модель distilled RoBerta base с LoRA адаптером может получить те же показатели метрик  для классификации текстов, написанных человеком, что и vanilla distilled RoBerta base на наборе данных с 4 классами.


\smallskip
\textbf{Ключевые слова}:  машинное обучение; линейная алгебра; аппроксимация матриц; уменьшение размерности пространств; классификация AI текстов; многоклассовая классификация текстов; большие языковые модели.
\end{abstract}