\documentclass[a4paper,14pt]{extarticle}
\usepackage[utf8]{inputenc}
\usepackage[T1,T2A]{fontenc}
\usepackage[english,russian]{babel}

\usepackage{amsthm}
\usepackage{graphicx}
\usepackage{caption}
\usepackage{amssymb}
\usepackage{amsmath}
\usepackage{mathrsfs}
\usepackage{euscript}
\usepackage{graphicx}
\usepackage{subfig}
\usepackage{caption}
\usepackage{color}
\usepackage{bm}
\usepackage{tabularx}
\usepackage{adjustbox}


\usepackage[toc,page]{appendix}

\usepackage{comment}
\usepackage{rotating}

\DeclareMathOperator*{\argmax}{arg\,max}
\DeclareMathOperator*{\argmin}{arg\,min}
\DeclareMathOperator{\E}{\mathbb{E}}

\newtheorem{theorem}{Теорема}
\newtheorem{lemma}[theorem]{Лемма}
\newtheorem{definition}{Определение}[section]

\numberwithin{equation}{section}

\newcommand*{\No}{No.}

\begin{document}

% Титульный лист
%\input{./frontpage.tex}

% Аннотация
\newpage


\begin{abstract}
В данной работе исследуется метод уменьшения размерности пространства обучаемых параметров в задаче детектирования AI текстов, задача многоклассовой классификации. Для дообучения использовалась дистилированная модель трансформер с двунаправленным кодированием представлений~(англ. DistilRoBERTa) с использованием низкорангового разложения~(англ. Low Rank Adaptation) матриц весов. Были проведены эксперименты для оценки эффективности использования LoRA адаптера для аппроксимации матрицы весов с точки зрения времени, ресурсов или точности. Было показано, что при меньших ресурсах модель DistilRoBERTa с LoRA адаптером может получить те же показатели метрик  для классификации текстов, написанных человеком, что и DistilRoBERTa без LoRA на наборе данных с 4 классами.


\smallskip
\textbf{Ключевые слова}:  машинное обучение; линейная алгебра; аппроксимация матриц; уменьшение размерности пространств; классификация AI текстов; многоклассовая классификация текстов; большие языковые модели.
\end{abstract}

% Нумерация должна начинаться со второй страницы
\setcounter{page}{2}

% Оглавление
\newpage
\tableofcontents

% Обозначения и сокращения
% \input{./dict.tex}

% Введение
\newpage


\section{Введение}
\paragraph{Актуальность темы.} 
    Уменьшение размерности пространства обучаемых параметров в задаче адаптации к домену упрощает процесс обучения и улучшает вычислительную эффективность. Путем сокращения количества параметров, которые необходимо обновить во время обучения, модель может потенциально быстрее сходиться и затрачивать меньше вычислительных ресурсов; это связано с структурой нейронной сети~--- ее сложность и потребление ресурсов зависят от количества обучаемых параметров. Уменьшение размерности может быть особенно важным в сценариях адаптации к домену, где обрабатываются большие объемы данных и происходит обучение с большим числом параметров.
    
    Методы, направленные на решение проблемы снижения размерности: метод главных компонент~\cite{wold1987principal} и его адаптации: тензорное разложение~\cite{oseledets2011tensor,qazi2024gpt}, каноническое полиадическое разложение~\cite{zare2018extension} выбирают наиболее важные векторы признаков из набора данных, используя сингулярное разложение матрицы для нахождения первых~$K$ собственных векторов с наибольшим собственным значением. Методы, осуществляющие отбор признаков: регуляризация LASSO (L1)~\cite{fonti2017feature}, оценка Фишера~\cite{gu2012generalized} или тест Хи-квадрат~\cite{zhai2018chi}. Метод снижения размерности, основанный на дообучении больших текстовых моделей~--- дистилляция~\cite{hsieh2023distilling}; в этом методе большая генеративная модель является \textit{учителем}, а меньшая~--- \textit{учеником}. Модель ученика обучают с использованием прогнозов \textit{учителя}. Эти идеи были впервые представлены в работах Дж. Хинтона~\cite{hinton2015nips} и В.Н. Вапника~\cite{vapnik2015learning}.

    Метод, рассмотренный в данной работе~--- низкоранговое разложение~(англ. Low Rank Adaptation)~\cite{hu2021lora}, который разработан на основе идеи о том, что предварительно обученные языковые модели имеют низкую внутреннюю размерность и могут эффективно обучаться, несмотря на проецирование на меньшее подпространство~\cite{aghajanyan2020intrinsic}. Данный метод, как и метод главных компонент, использует сингулярное разложение матрицы для нахождения низкоранговых приближений матрицы весов. LoRA используется для решения различных проблем seq2seq, таких как:~\cite{zhang2023adding,dettmers2024qlora,dai2024instructblip}. Этот подход особенно популярен в задачах преобразования видео в текст, так как этим задачам свойственны вариативность распределения входных данных и разнообразие задач, обусловленные дополнительным визуальным входным данным~\cite{dai2024instructblip}.

    В данной работе метод LoRA применяется к задаче обнаружения текстов, написанных искуственным интеллектом или человеком.
    Задача обнаружения текстов, написанных искусственным интеллектом стала особенно популярна с выходом новых больших моделей от OpenAI и Google~\cite{OpenAI, Google}, так как определить кем написан тот или иной текст все сложнее~\cite{anderson2023ai, weber2023testing}. В данной статье мы работали над дообучением популярной модели RoBerta с использованием LoRA адаптеров с целью понижения размерности пространства обучаемых параметров. Предполагается, что LoRA может быть так же эффективен в решении задач классификации, как и в задачах генерации: LoRA доказала свою эффективность во многих задачах генерации~\cite{dettmers2024qlora,hu2021lora,dai2024instructblip}.

Как отмечено в работах~\cite{he2023mgtbench,abdali2024decoding}, все подходы к решению задачи классификации ai текстов разделяются на два типа: ориентированные на анализ признакового пространства и ориентированные на дообучение моделей. 
 Анализ признакового пространства основывается на извлечении и анализе характеристик текста~--- лексических, синтаксических, семантических или стилистических характеристик:~\cite{liang2023gpt,yu2023gpt,yang2023dna}.
 Дообучение больших языковых моделей основывается на изучении параметров и возможностей модели и последующем дообучении модели к задаче классификации текстов:~\cite{wolf2019huggingface,wolf2019huggingface,qasim2022fine}.


\textbf{Цели работы.}
Исследуются методы снижения пространства обучаемых параметров при помощи сингулярного разложения матриц, а также кооректность применения изучаемых методов к задаче классификации текстов. 

\textbf{Методы исследования.}
Применяется низкоранговое разложение~(англ. Low Rank Adaptation) к матрицам параметров в больших языковых моделях. Используются статистические методы оценки функции минимизации функции потерь, а также свойтва матриц для доказательства применимости предложенного метода к задаче классификации.

\textbf{Теоретическая значимость.}
В работе проведен теоретический анализ проблемы снижения размерности пространства обучаемых параметров. Доказана теорема об применимости модели BERT~\cite{vaswani2017attention} с адаптером LoRA к задаче многоклассовой классификации. 

\textbf{Практическая значимость.}
Проведен вычислительный эксперимент, показывающий улучшение качества и экономию ресурсов при решении задачи классификации текстов.


% Постановка задачи классификации текстов
\newpage


\section{Постановка задачи классификации текстов}
Модель семейства трансформеров решает проблему генерации последовательность в последовательность(англ. seq2seq), которая формулируется следующим образом согласно~\cite{thickstun2021transformer}: 
\begin{align}
f_\theta: \mathbb{R}^{n \times d} \rightarrow \mathbb{R}^{n \times d}   
\end{align}
где модель $f_{\theta}$ отображает неупорядоченный набор из $n$ элементов в $\mathbb{R}^{d}$ в другой неупорядоченный набор из $n$ элементов в $\mathbb{R}^{d}$. Требуется переформулировать задачу в классификатор текстов, определяя, были ли они написаны человеком или одной из трех предложенных языковых моделей.
\begin{align}
f_\theta : \hat{V} \rightarrow [N_c]
\end{align}
где $\hat{V} \subset V^{}$; $V$~--- словарь токенов и $V^{}$~--- его замыкание или множество всех текстов над $V$, $[N_c]$ - множество классов. Таким образом, модель отображает текст из $\hat{V}$ в класс из $[N_c]$.

Тогда $(X_i, c_i) \in V^{*} \times [N_c]$ для $i \in [N_{data}]$ является парой текст-класс, выбранной из $P(X, c)$, где $X_i$ - входной текст, а $c_i$~--- его класс. Таким образом, наша цель - оценить $P(c|X)$.

Согласно~\cite{hu2021lora} при дообучении модель инициализируется предварительно обученными весами $\Phi_0$ и обновляется до $\Phi_0 + \delta\Phi$. Тогда задача минимизации функции потерь имеет вид:
\begin{equation}
\label{eq:12} 
\begin{aligned}
\min _{\Phi} -\sum_{X_i \in \hat{V} \subset V^{*}} \sum_{c_i \in [N_c]} \log \left(P_{\Phi}\left(c_i \mid X_i\right)\right) =\\ 
= \max _{\Phi} \sum_{X_i \in \hat{V} \subset V^{*}} \sum_{c_i \in [N_c]} \log \left(P_{\Phi}\left(c_i \mid X_i\right)\right)
\end{aligned}
\end{equation}
Здесь $\mid\Delta\Phi\mid = \mid\Phi_0\mid$.\\ В то время как при использовании LoRA: $\Delta\Phi = \Delta\Phi(\Theta)$, где $\Theta$ - набор параметров намного меньшего размера, $\mid\Theta\mid \ll \mid\Phi_0\mid$ и задача минимизации функции потерь имеет вид:
\begin{equation}
\label{eq:13}
\begin{aligned}
\min _{\Theta} -\sum_{X_i \in \hat{V} \subset V^{*}} \sum_{c_i \in [N_c]} \log \left(P_{\Phi_0+\Delta \Phi(\Theta)}\left(c_i \mid X_i\right)\right) =\\
= \max _{\Theta} \sum_{X_i \in \hat{V} \subset V^{*}} \sum_{c_i \in [N_c]} \log \left(P_{\Phi_0+\Delta \Phi(\Theta)}\left(c_i \mid X_i\right)\right) 
\end{aligned}
\end{equation}




% Предложенный метод
\newpage

\section{Предложенный метод и его корректность}
Опираясь на оригинальную статью LoRA~\cite{hu2021lora}, в этой работе LoRA была применина к задаче классификации. Структура LoRA адаптера:
\begin{center}
\begin{tabular}{ | c | c| } 
 \hline
  Fine tuning & LoRA fine tuning\\ 
 \hline
 $W_{upd} = W + \Delta W$ & $W_{upd} = W + AB$\\ 
 $\hat{y} = xW_{upd}= x(W + \Delta W)$ & $\hat{y} = xW_{upd}= x(W + AB)$\\
 $\hat{y} = xW + x\Delta W$ & $\hat{y} = xW + xAB$ \\
 \hline
\end{tabular}
\end{center}
Где $W \in \mathbb{R}^{d \times k}$ - предобученные веса, $\Delta W \in \mathbb{R}^{d \times k}$ - матрица обновленных весов. $\Delta W$ is приближается с помощью метода LoRA произведением $A \cdot B$, где $A \in \mathbb{R}^{d \times r}$, $B \in \mathbb{R}^{r \times k}$ и $r$ - гиперпараметр ранга. \newline
Здесь $A \sim \mathcal{N}(0,\,\sigma^{2})$ и $B = [0]_{r \times k}$.


\textit{Докажем состоятельность предложенной модели}\\
Сходимость традиционной модели трансформер была доказана в работе~\cite{lee2023mathematical}. Доказательство приведено для задачи классификации:

\subsection{Теорема 1~\cite{lee2023mathematical}}
\begin{theorem}
Будем считать, что: 
\begin{itemize}
    \item Существует модель с набором параметров $\theta^*$, которая может аппроксимировать достоверное распределение, сгенерированное данными, с минимальным расхождением по KL-дивергенции: 
    \begin{equation}
    \label{eq:1}
    \exists \theta^*: \theta^* = \argmin _\theta D_{KL}(P_{true} \mid\mid P_{model}(\cdot, \theta))
    \end{equation}
     \item По мере увеличения размера набора данных $\hat{V}$ эмпирическое распределение приближается к истинному распределению, генерирующему данные.
     \item $\mathscr{L}(\theta)$ - непрерывная, дифференцируемая. Где
\end{itemize} 
\begin{equation}
\label{eq:2}
\mathscr{L}(\theta) = -\sum_{X_i \in \hat{V} \subset V^{*}} \sum_{c_i \in [N_c]} \log \left(P_{\Phi_0+\Delta \Phi(\theta)}\left(c_i \mid X_i\right)\right)
\end{equation}

Тогда минимизация $\mathscr{L}(\theta)$ приводит к получению оценки истинного распределения, генерирующего данные.

\end{theorem}
\renewcommand\qedsymbol{$\blacksquare$}
\begin{proof}
Докажем сходимость следующей функции:

Согласно минимизации функции эмпирического риска, минимум $\mathscr{L}(\theta)$ приближается к минимуму матожидания риска, поскольку размер $\hat{V}$ стремится к бесконечности.
\begin{equation}
\label{eq:3}
\begin{aligned}
\lim_{\mid\hat{V}\mid\to\infty} \argmin _\theta \E \mathscr{L}(\theta) =\\
\argmin _\theta \E _{X_i\in P_{true}}[\sum_{c_i \in [N_c]} \log \left(P_{\Phi_0+\Delta \Phi(\theta)}\left(c_i \mid X_i\right)\right] =\\
\argmin _\theta D_{KL}(P_{true}\mid\mid P_{model}(\theta)) = \theta^*
\end{aligned}
\end{equation}

Равенство~\eqref{eq:3} верно в силу равномерной сходимости $\mathscr{L}(\theta)$ и в силу определения KL-дивергенции.
\begin{equation}
\label{eq:4}
D_{KL}(P || Q)=\int_{-\infty}^{\infty} p(x) \log \left(\frac{p(x)}{q(x)}\right) dx = \E _{x \sim p(x)}[\log \left(\frac{p(x)}{q(x)}\right)]
\end{equation}
\end{proof}

 Докажем, что LoRA применима к задаче классификации. LoRA  используется для решения различных проблем seq2seq, таких как:\\~\cite{zhang2023adding},~\cite{dettmers2024qlora},~\cite{dai2024instructblip}. Этот подход особенно популярен в задачах преобразования видео в текст, так как им(задачам) свойственны "богатые распределения входных данных и разнообразие задач, обусловленные дополнительным визуальным входным данным"~\cite{dai2024instructblip}. 

В то же время для решения задачи классификации с помощью BERT~\cite{vaswani2017attention} требуется не более чем дополнительный softmax слой после BERT~\cite{sun2019fine}: 
\begin{equation}
\label{eq:5}
\begin{aligned}
p(c \mid \mathbf{x})=\operatorname{softmax}(W^T \mathbf{x})\\
\hat{\mathbf{y}}=\operatorname{softmax}\left(W^T \mathbf{x}\right)=\frac{\exp \left(W^T \mathbf{x}\right)}{\sum_{i=1}^k \exp \left(W^T \mathbf{x}\right)_i}
\end{aligned}
\end{equation}
Где $\mathbf{x}$ - это выходной результат последнего слоя BERT.\\
Структура BERT:

\begin{figure}[h]
    \centering
    \includegraphics[width=1.0\textwidth]{images/bert_architecture copy.pdf}
    \caption{BERT structure}
\end{figure}

\newpage
Где, согласно~\cite{thickstun2021transformer},\\
Attention:
\begin{equation}
\begin{aligned}
Q^{(h)}\left(\mathbf{x}_i\right)=W_{h, q}^T \mathbf{x}_i, \quad K^{(h)}\left(\mathbf{x}_i\right)=W_{h, k}^T \mathbf{x}_i, \\ V^{(h)}\left(\mathbf{x}_i\right)=W_{h, v}^T \mathbf{x}_i, \quad W_{h, q}, W_{h, k}, W_{h, v} \in \mathbb{R}^{d \times k} 
\end{aligned}
\end{equation}
\begin{equation}
\begin{aligned}
\alpha_{i,j}^{(h)}=\operatorname{softmax}_j\left(\frac{\left\langle Q^{(h)}\left(\mathbf{x}_i\right), K^{(h)}\left(\mathbf{x}_j\right)\right\rangle}{\sqrt{k}}\right), \\
\mathbf{u}_i^{\prime}=\sum_{h=1}^H W_{c, h}^T \sum_{j=1}^n \alpha_{i, j}^{(h)} V^{(h)}\left(\mathbf{x}_j\right), \\
\end{aligned}
\end{equation}
LayerNorm,
Feed Forward Network,
LayerNorm:
\begin{equation}
\begin{aligned}
&\mathbf{u}_i=\operatorname{LayerNorm}\left(\mathbf{x}_i+\mathbf{u}_i^{\prime} ; \gamma_1, \beta_1\right), \\
& \mathbf{z}_i^{\prime}=W_2^T \operatorname{ReLU}\left(W_1^T \mathbf{u}_i\right), \\
& \mathbf{z}_i=\text { LayerNorm }\left(\mathbf{u}_i+\mathbf{z}_i^{\prime} ; \gamma_2, \beta_2\right) \text {, }
\end{aligned}
\end{equation}
Layer normalization может быть переписана в соответствии с оригинальной статьей~\cite{ba2016layer}:
\begin{equation}
\begin{aligned}
& \operatorname{LayerNorm}(\mathbf{z} ; \gamma, \beta)=\gamma \frac{\left(\mathbf{z}-\mu_{\mathbf{z}}\right)}{\sigma_{\mathbf{z}}}+\beta, \\
& \gamma, \beta \in \mathbb{R}^k . \\
& \mu_{\mathbf{z}}=\frac{1}{k} \sum_{i=1}^k \mathbf{z}_i, \quad \sigma_{\mathbf{z}}=\sqrt{\frac{1}{k} \sum_{i=1}^k\left(\mathbf{z}_i-\mu_{\mathbf{z}}\right)^2} . \\
&
\end{aligned}
\end{equation}

\subsection{Теорема 2}
\begin{theorem}
В рамках задачи классификации, при заданных условиях:
\begin{itemize}
    \item Модель семейства Bert с указанной выше математической структурой и дополнительным слоем 
    \begin{equation}
    \label{eq:6}
        \hat{\mathbf{y}}=\operatorname{softmax}\left(W_{upd}^T \mathbf{x}\right)=\frac{\exp \left(W_{upd}^T \mathbf{x}\right)}{\sum_{i=1}^k \exp \left(W_{upd}^T \mathbf{x}\right)_i}
    \end{equation}
    где 
    \begin{equation}
    \label{eq:7}
    W_{upd} =\underset{(d \times k) }{W} + \underset{(d \times k)}{\Delta W}
    \end{equation}
    и $x$ - это выходной результат BERT, $W$ - матрица весов, $\Delta W$ - матрица обновленных весов.
    \item  Данная модель Bert без дополнительного слоя также корректно работает с аппроксимацией 
    \begin{equation}
    \label{eq:8}
    \underset{(d \times k)}{\Delta W} = \underset{(d \times r)}{ A} \times \underset{(r \times k)}{B}
    \end{equation}
    \item Выполняется lemma 1.0.1. (можно считать данную модель состоятельной).
\end{itemize}
Тогда можно утверждать, что при~\eqref{eq:10} заданная модель BERT с дополнительным слоем гарантирует корректную выходную матрицу.
\end{theorem}
\renewcommand\qedsymbol{$\blacksquare$}
\begin{proof} 
Докажем, что выход из дополнительного слоя корректен:\\ По дистрибутивному свойству сложения матриц и ассоциативному свойтсву произведения матриц: 
\begin{equation}
\label{eq:9}
\begin{aligned}
\hat{\mathbf{y}} = \operatorname{softmax}\left(W_{upd}^T \mathbf{x}\right) =
\operatorname{softmax}\left((W + \Delta W)^T \mathbf{x}\right) =\\ \\
= \frac{\exp \left(W^T \mathbf{x} + \Delta W^T \mathbf{x}\right)}{\sum_{i=1}^k \exp \left(W^T \mathbf{x} + \Delta W^T \mathbf{x}\right)_i}=\\ \\
= \frac{\exp \left(W^T \mathbf{x}\right) \exp \left(\Delta W^T \mathbf{x}\right)}{\sum_{i=1}^k \exp \left(W^T \mathbf{x}\right)_i \exp \left(\Delta W^T \mathbf{x}\right)_i}
\end{aligned}
\end{equation} 
где $\mathbf{x}$ - выходная матрица последнего слоя Bert.
$\mathbf{x}$ корректна по условию.\\
В предложенной модели с использованием LoRA:
\begin{equation}
\label{eq:10}
\begin{aligned}
\hat{\mathbf{y}} = \operatorname{softmax}\left(W_{upd} \mathbf{x}\right) =
\operatorname{softmax}\left((W + AB)^T \mathbf{x}\right) =\\ \\
= \frac{\exp \left(W^T \mathbf{x} + (AB)^T \mathbf{x}\right)}{\sum_{i=1}^k \exp \left(W^T \mathbf{x} + (AB)^T \mathbf{x}\right)_i}=\\ \\
= \frac{\exp \left(W^T \mathbf{x}\right) \exp \left((AB)^T \mathbf{x}\right)}{\sum_{i=1}^k \exp \left(W^T \mathbf{x}\right)_i \exp \left((AB)^T \mathbf{x}\right)_i}
\end{aligned}
\end{equation} 
где $\mathbf{x}$ также выходная матрица BERT с LoRA и $\mathbf{x}$ корректна по условию. \newline
Так как финальные размерности остались неизменными, как и $W^T\mathbf{x}$, легко заметить: 
\begin{equation}
\label{eq:11}
\begin{aligned}
 \underset{(k \times d)}{\Delta W^T} \times \mathbf{x} = u\\
(\underset{(d \times r)}{A} \times \underset{(r \times k)}{B})^T \times \mathbf{x} = (\underset{(k \times r)}{B^T} \times \underset{(r \times d)}{A^T}) \times \mathbf{x} = u^*
\end{aligned}
\end{equation}
Так как~\eqref{eq:10}, можно заключить что ${u^*} = {u}$ и является корретной матрицей, а следовательно предложенная модель работает корректно.
\end{proof}


% Рубрика эксперименты
\newpage

\section{Вычислительный эксперимент}
\subsection{Данные} 
Открытый исходный датасет для мультиклассовой классификации текстов, написанных человеком и различными языковыми моделями~\cite{semeval2024task8}. Изначально было представлено 6 классов (включая человека), но из-за технических ограничений количество классов было сокращено до 4: ChatGPT, Davinci, Cohere, Humans. Всего в датасете 21 000 текстов с разметкой по классам.

\subsection{Предобработка}
Тексты были токенизированы при помощи AutoTokenizer~\cite{wolf2019huggingface}. Дополнительная обработка не требуется из-за структуры модели~\cite{vaswani2017attention}.

\subsection{Эксперименты} 
Для всех экспериментов использовалась предобученная модель DistilRoBERTa base (далее в тексте - DRoBERTa-base)~\cite{liu2019roberta}. Модель использовалась со следующими гиперпараметрами: доля тренировочного/тестового набора данных - 0.9/0.1; 3 эпохи обучения. Для экспериментов с использованием алгоритма LoRA были использованы все вышеуказанные параметры, а также ранг матриц аппроксимации r = 5.

\newpage
\subsection{Эксперимент (1)}
\textbf{Предобученная модель DRoBERTa-base обучилась на всем обучающем датасете.}\\ \\
После обучения для оценки использовались матрица ошибок и метрики точности, полноты и F1-меры: \newline
\textbf{train\_runtime: 4041.3188}
\begin{figure}[h]
    \centering
    \includegraphics[width=.75\textwidth]{images/bert_vanilla.png}
    \includegraphics[width=.65\textwidth]{images/exp1_report.png}
    \caption{результат эксперимента 1}
    \label{fig:1}
\end{figure}


\newpage
\subsection{Эксперимент (2)}
\textbf{Предобученная модель DRoBERTa-base с использованием LoRA обучилась на всем обучающем датасете.}\\ \\
\begin{text}
 trainable params: 685828 all params: 82807304 || trainable\%: 0.8282    
\end{text}
Только 0.828\% параметров обучаются при использовании LoRA\\ 
\textbf{train\_runtime: 3210.977}\\
\begin{figure}[h]
    \centering
    \includegraphics[width=.75\textwidth]{images/bert_lora_4.png}
    \includegraphics[width=.65\textwidth]{images/exp2_report.png}
    \caption{результат эксперимента 2}
    \label{fig:2}
\end{figure}

\newpage
\subsection{Эксперимент (3)}
\textbf{Три независимые модели DRoBERTa-base с исп. LoRA обучались на парах классов: GPT vs Human, Davinci vs Human, Cohere vs Human.}\\ \\
\textbf{ChatGPT vs Human}\\
\textbf{train\_runtime: 1633.8114}
\begin{figure}[h]
    \centering
    \includegraphics[width=.75\textwidth]{images/chatGPT_human.png}
    \includegraphics[width=.65\textwidth]{images/exp3_gpt.png}
    \caption{результат эксперимента 3.1}
    \label{fig:3}
\end{figure}

\newpage
\textbf{Cohere vs Human}\\
\textbf{train\_runtime: 1583.556}
\begin{figure}[h]
    \centering
    \includegraphics[width=.75\textwidth]{images/cohere_human.png}
    \includegraphics[width=.65\textwidth]{images/exp3_cohere.png}
    \caption{результат эксперимента 3.2}
    \label{fig:4}
\end{figure}

\newpage
\textbf{Davinci vs Human}\\
\textbf{train\_runtime: 1632.395}
\begin{figure}[h]
    \centering
    \includegraphics[width=.75\textwidth]{images/davinci_human.png}
    \includegraphics[width=.65\textwidth]{images/exp3_davinci.png}
    \caption{результат эксперимента 3.3}
    \label{fig:5}
\end{figure}








% Выводы
\newpage

\section{Заключение}
В работе рассматривался метод LoRA снижения размерности пространства обучаемых параметров в задаче классификации текстов, написанных большими языковыми моделями. Была сформулированна и доказана теорема о конструктивности предложенного метода.

В ходе эксперимента на датасете из текстов, написанных как языковыми моделями, так и человеком, была доказана эффективность предложенного метода.
При решении задачи мультиклассовой классификации предложенная модель BERT \& LoRA  тратит меньше ресурсов, чем модель без использования LoRA, но метрики качества падают. Однако, при решении тремя одинаковыми независимыми моделями задачи бинарной классификации с последующим усреднением метрики качетва растут, а использование ресурсов~--- нет. Таким образом, в данной работе теоритически и экспериментально доказана состоятельность и эффективность предложенного метода.



% Библиографические ссылки
\newpage
\bibliographystyle{plain}
\bibliography{ref} 

% Приложения
%\newpage


\begin{lemma}
\label{statement:1}
Утверждение из статьи~\cite{donini2018empirical}:
В терминах постаноки задачи классификации текстов:
для функции риска $L(\Theta)$:
\begin{equation}
\label{eq:2.3}
L(\Theta) = \E _{X_i} \mathscr{L}(X_i; \Theta)
\end{equation}
и функции эмпирического риска $\hat{L}(\Theta)$:
\begin{equation}
\hat{L}(\Theta) = \frac{1}{\mid \hat{V} \mid} \sum_{X_i \in \hat{V}} \mathscr{L}(X_i; \Theta) 
\end{equation}
\text{верно следующее:}
\begin{equation}
\sup _{\Theta} \mid L(\Theta) - \hat{L}(\Theta) \mid \le \delta(\mid \hat{V} \mid):
\end{equation}
\begin{equation}
\delta(\mid \hat{V} \mid) \xrightarrow[\mid \hat{V} \mid \rightarrow \inf]{} 0
\end{equation}
\end{lemma}


\end{document}